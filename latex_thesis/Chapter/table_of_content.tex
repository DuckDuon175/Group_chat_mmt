
%mục lục
\phantomsection\addtocontents{toc}{\protect\thispagestyle{empty}}
\tableofcontents % tạo mục lục tự động
\thispagestyle{empty}
\cleardoublepage

\pagenumbering{roman} %đánh số thứ tự la mã
\phantomsection\section*{DANH MỤC KÝ HIỆU VÀ CHỮ VIẾT TẮT}%dấu * để không đánh số
\addcontentsline{toc}{section}{\numberline{}DANH MỤC KÝ HIỆU VÀ CHỮ VIẾT TẮT} % LƯU VÀO TRONG MỤC LỤC



\begin{table}[H]
  \centering
  \begin{tabularx}{0.85\textwidth}{
  | >{\centering\arraybackslash}m{3cm}
  | >{\centering\arraybackslash}X|
  }
  \hline
  \bfseries Từ viết tắt     &\bfseries Thay cho\hspace{1cm}\\ \hline
  Grad-CAM     &\\ \hline
  SGD  &Stochastic Gradient Descent\\ \hline
  CNN   &Convolutional Neural Network\\ \hline
  DNN   &Deep Neural Network\\ \hline
  RNN    &Recurrent Neural Network\\ \hline
  FC   &Fully-Connected\\ \hline
  RGB     &Red, Green, Blue\\ \hline
  ML  &Machine Learning\\ \hline
  CGAP   &Cumulative Global Average Pooling\\ \hline
  SE   &Squeeze-and-Excitation\\ \hline
  NAS    &Neural Architecture Search\\ \hline
  BN   &Batch Normalization\\ \hline
  FLOPs   &Floating point operations per second\\ \hline
  \end{tabularx}
  \label{bang31}
\end{table}


\cleardoublepage

{
\let\oldnumberline\numberline
\renewcommand{\numberline}{\figurename~\oldnumberline}%
\listoffigures
} %tạo danh mục hình vẽ tự động
\phantomsection\addcontentsline{toc}{section}{\numberline{}DANH MỤC KÝ HÌNH VẼ}
\cleardoublepage

{
\let\oldnumberline\numberline
\renewcommand{\numberline}{\tablename~\oldnumberline}%
\listoftables
} %tạo danh mục hình vẽ tự động %tạo danh mục bảng biểu
\phantomsection\addcontentsline{toc}{section}{\numberline{}DANH MỤC BẢNG BIỂU}
\cleardoublepage
