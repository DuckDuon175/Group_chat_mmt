
\newpage \pagestyle{empty}
\begin{center}
    \textbf{\fontsize{14pt}{0pt}\selectfont ĐÁNH GIÁ QUYỂN ĐỒ ÁN TỐT NGHIỆP}\\
    \vspace{10pt}
    \fontsize{14pt}{0pt}\selectfont (Dùng cho giản viên hướng dẫn) 
\end{center}
\vspace{14pt}
\fontsize{13pt}{20pt}\selectfont Tên giảng viên đánh giá:\\
\fontsize{13pt}{20pt}\selectfont Họ và tên Sinh Viên:
\hspace{5.5cm}
\fontsize{13pt}{20pt}\selectfont MSSV:\\
\fontsize{13pt}{20pt}\selectfont Tên đồ án:\\
\fontsize{13pt}{20pt}\selectfont \\
\textbf{\fontsize{13pt}{20pt}\selectfont Chọn các mức điểm phù hợp cho sinh viên trình bày theo các tiêu chí dưới đây:}\\
\fontsize{13pt}{20pt}\selectfont Rất kém (1); Kém (2); Đạt (3); Giỏi (4); Xuất sắc (5)
\begin{table}[H]
    \fontsize{11}{11}\selectfont
    \begin{tabular}{|M{1cm}|M{11cm}|M{0.3cm}|M{0.3cm}|M{0.3cm}|M{0.3cm}|M{0.3cm}|}
    \hline
    \rowcolor[rgb]{0.52,0.96,1}
    \multicolumn{7}{|p{1.01\linewidth}|}{\textbf{Có sự kết hợp giữa lý thuyết va thực hành (20)}} \\
    \hline
    1 &  \raggedright Nêu rõ tính cấp thiết và quan trọng của đề tài, các cấn đề và các giả thuyết (bao gồm mục đíc và tính phù hợp) cũng như phạm vi ứng dụng của đồ án  & 1 & 2 & 3 & 4 & 5\\
    \hline
    2 & \raggedright Cập nhật kết quả nghiên cứu gần đây nhất (trong nước/quốc tế) & 1 & 2 & 3 & 4 & 5\\
    \hline
    3 & \raggedright Nêu rõ và chi tiết phương pháp nghiên cứu/giải quyết vấn đề & 1 & 2 & 3 & 4 & 5\\
    \hline
    4 & \raggedright Có kết quả mô phỏng/thực nghiệm và trình bày rõ ràng kết quả đo được & 1 & 2 & 3 & 4 & 5\\
    \hline
    
    \rowcolor[rgb]{0.52,0.96,1}
    \multicolumn{7}{|p{1.01\linewidth}|}{\textbf{Có khả năng phân tích và đánh giá kết quả (15)}} \\
    \hline
    5 &  \raggedright Kế hoạch làm việc rõ ràng bao gồm mục tiêu và phương pháp thực hiện dựa trên kết quả nghiên cứu lý thuyết một cách có hệ thống  & 1 & 2 & 3 & 4 & 5\\
    \hline
    6 & \raggedright Kết quả được trình bày một cách logic và dễ hiểu, tất cả kết quả đều được phân tích và đánh giá thỏa đáng. & 1 & 2 & 3 & 4 & 5\\
    \hline
    7 & \raggedright Trong phần kết luận, tác giả chỉ rõ sự khác biệt (nếu có) giữa kết quả đạt được và mục tiêu ban đầu đề ra đồng thời cung cấp lập luận để đề xuất hướng giải quyết có thể thực hiện trong tương lai. & 1 & 2 & 3 & 4 & 5\\
    \hline
    
    \rowcolor[rgb]{0.52,0.96,1}
    \multicolumn{7}{|p{1.01\linewidth}|}{\textbf{Kỹ năng viết quyển đồ án (10)}} \\
    \hline
    8 &  \raggedright Đồ án trình bày đúng mẫu quy định với cấu trúc các chương logic và đẹp mắt (bảng biểu, hình ảnh rõ ràng, có tiêu đề, được đánh số thứ tự và được giải thích hay đề cập đến trong đồ án, có căn lề, dấu cách sau dấu chấm, dấu phẩy v.v), có mở đầu chương và kết luận chương, có liệt kê tài liệu tham khảo và có trích dẫn đúng quy địnhg  & 1 & 2 & 3 & 4 & 5\\
    \hline
    9 & \raggedright Kỹ năng viết xuất sắc (cấu trúc câu chuẩn, văn phong khoa học, lập luận logic và có cơ sở, từ vựng sử dụng phù hợp v.v.). & 1 & 2 & 3 & 4 & 5\\
    \hline
    
    \rowcolor[rgb]{0.52,0.96,1}
    \multicolumn{7}{|p{1.01\linewidth}|}{\textbf{Thành tựu nghiên cứu khoa học (5) (chọn 1 trong 3 trường hợp)}} \\
    \hline
    10a &  \raggedright Có bài báo khoa học được đăng hoặc chấp nhận đăng/đạt giải SVNC khoa học giải 3 cấp  Viện trở lên/các giải thưởng khoa học (quốc tế/trong nước) từ giải 3 trở lên/ Có đăng ký bằng phát minh sáng chế  & 1 & 2 & 3 & 4 & 5\\
    \hline
    10b & \raggedright Được báo cáo tại hội đồng cấp Viện trong hội nghị sinh viên nghiên cứu khoa học nhưng không đạt giải từ giải 3 trở lên/Đạt giải khuyến khích trong các kỳ thi quốc gia và quốc tế khác về chuyên ngành như TI contest.). & 1 & 2 & 3 & 4 & 5\\
    \hline
    10c & \raggedright Không có thành tích về nghiên cứu khoa học & 1 & 2 & 3 & 4 & 5\\
    \hline
    \rowcolor[rgb]{0.52,0.96,1}
    \multicolumn{2}{|p{0.776\linewidth}|}{\textbf{Điểm tổng}} & \multicolumn{5}{|p{0.205\linewidth}|}{\textbf{\hspace{2cm}/50}} \\
    \hline
    \rowcolor[rgb]{0.52,0.96,1}
    \multicolumn{2}{|p{0.776\linewidth}|}{\textbf{Điểm tổng quy đổi về thang 10}}&\multicolumn{5}{|p{0.205\linewidth}|}{} \\
    \hline
    \end{tabular}
    \label{mul_table}
\end{table}
\raggedright\textbf{\itshape\fontsize{13pt}{20pt}\selectfont Nhận xét khác} \fontsize{13pt}{20pt}\selectfont (về thái độ và tinh thần làm việc của sinh viên)
\newline



\vspace{5cm}
\hspace{9cm}Hà Nội, ngày\hspace{0.5cm}tháng\hspace{0.5cm}năm

\hspace{10cm}\textbf{Người nhận xét}
\vspace{2cm}
\hspace{9.5cm} (Ký và ghi rõ họ tên)
\newpage
\begin{center}
    \textbf{\fontsize{14pt}{0pt}\selectfont ĐÁNH GIÁ QUYỂN ĐỒ ÁN TỐT NGHIỆP}\\
    \vspace{10pt}
    \fontsize{14pt}{0pt}\selectfont (Dùng cho cán bộ phản biện) 
\end{center}
\vspace{14pt}
\fontsize{13pt}{20pt}\selectfont Tên giảng viên đánh giá:\\
\fontsize{13pt}{20pt}\selectfont Họ và tên Sinh Viên:
\hspace{5.5cm}
\fontsize{13pt}{20pt}\selectfont MSSV:\\
\fontsize{13pt}{20pt}\selectfont Tên đồ án:\\

\vspace{0.8cm}

\textbf{\fontsize{13pt}{20pt}\selectfont Chọn các mức điểm phù hợp cho sinh viên trình bày theo các tiêu chí dưới đây:}\\
\fontsize{13pt}{20pt}\selectfont Rất kém (1); Kém (2); Đạt (3); Giỏi (4); Xuất sắc (5)
\begin{table}[H]
    \fontsize{11}{11}\selectfont
    \begin{tabular}{|M{1cm}|M{11cm}|M{0.3cm}|M{0.3cm}|M{0.3cm}|M{0.3cm}|M{0.3cm}|}
    \hline
    \rowcolor[rgb]{0.52,0.96,1}
    \multicolumn{7}{|p{1.01\linewidth}|}{\textbf{Có sự kết hợp giữa lý thuyết va thực hành (20)}} \\
    \hline
    1 &  \raggedright Nêu rõ tính cấp thiết và quan trọng của đề tài, các cấn đề và các giả thuyết (bao gồm mục đíc và tính phù hợp) cũng như phạm vi ứng dụng của đồ án  & 1 & 2 & 3 & 4 & 5\\
    \hline
    2 & \raggedright Cập nhật kết quả nghiên cứu gần đây nhất (trong nước/quốc tế) & 1 & 2 & 3 & 4 & 5\\
    \hline
    3 & \raggedright Nêu rõ và chi tiết phương pháp nghiên cứu/giải quyết vấn đề & 1 & 2 & 3 & 4 & 5\\
    \hline
    4 & \raggedright Có kết quả mô phỏng/thực nghiệm và trình bày rõ ràng kết quả đo được & 1 & 2 & 3 & 4 & 5\\
    \hline
    
    \rowcolor[rgb]{0.52,0.96,1}
    \multicolumn{7}{|p{1.01\linewidth}|}{\textbf{Có khả năng phân tích và đánh giá kết quả (15)}} \\
    \hline
    5 &  \raggedright Kế hoạch làm việc rõ ràng bao gồm mục tiêu và phương pháp thực hiện dựa trên kết quả nghiên cứu lý thuyết một cách có hệ thống  & 1 & 2 & 3 & 4 & 5\\
    \hline
    6 & \raggedright Kết quả được trình bày một cách logic và dễ hiểu, tất cả kết quả đều được phân tích và đánh giá thỏa đáng. & 1 & 2 & 3 & 4 & 5\\
    \hline
    7 & \raggedright Trong phần kết luận, tác giả chỉ rõ sự khác biệt (nếu có) giữa kết quả đạt được và mục tiêu ban đầu đề ra đồng thời cung cấp lập luận để đề xuất hướng giải quyết có thể thực hiện trong tương lai. & 1 & 2 & 3 & 4 & 5\\
    \hline
    
    \rowcolor[rgb]{0.52,0.96,1}
    \multicolumn{7}{|p{1.01\linewidth}|}{\textbf{Kỹ năng viết quyển đồ án (10)}} \\
    \hline
    8 &  \raggedright Đồ án trình bày đúng mẫu quy định với cấu trúc các chương logic và đẹp mắt (bảng biểu, hình ảnh rõ ràng, có tiêu đề, được đánh số thứ tự và được giải thích hay đề cập đến trong đồ án, có căn lề, dấu cách sau dấu chấm, dấu phẩy v.v), có mở đầu chương và kết luận chương, có liệt kê tài liệu tham khảo và có trích dẫn đúng quy địnhg  & 1 & 2 & 3 & 4 & 5\\
    \hline
    9 & \raggedright Kỹ năng viết xuất sắc (cấu trúc câu chuẩn, văn phong khoa học, lập luận logic và có cơ sở, từ vựng sử dụng phù hợp v.v.). & 1 & 2 & 3 & 4 & 5\\
    \hline
    
    \rowcolor[rgb]{0.52,0.96,1}
    \multicolumn{7}{|p{1.01\linewidth}|}{\textbf{Thành tựu nghiên cứu khoa học (5) (chọn 1 trong 3 trường hợp)}} \\
    \hline
    10a &  \raggedright Có bài báo khoa học được đăng hoặc chấp nhận đăng/đạt giải SVNC khoa học giải 3 cấp  Viện trở lên/các giải thưởng khoa học (quốc tế/trong nước) từ giải 3 trở lên/ Có đăng ký bằng phát minh sáng chế  & 1 & 2 & 3 & 4 & 5\\
    \hline
    10b & \raggedright Được báo cáo tại hội đồng cấp Viện trong hội nghị sinh viên nghiên cứu khoa học nhưng không đạt giải từ giải 3 trở lên/Đạt giải khuyến khích trong các kỳ thi quốc gia và quốc tế khác về chuyên ngành như TI contest.). & 1 & 2 & 3 & 4 & 5\\
    \hline
    10c & \raggedright Không có thành tích về nghiên cứu khoa học & 1 & 2 & 3 & 4 & 5\\
    \hline
    \rowcolor[rgb]{0.52,0.96,1}
    \multicolumn{2}{|p{0.776\linewidth}|}{\textbf{Điểm tổng}} & \multicolumn{5}{|p{0.205\linewidth}|}{\textbf{\hspace{2cm}/50}} \\
    \hline
    \rowcolor[rgb]{0.52,0.96,1}
    \multicolumn{2}{|p{0.776\linewidth}|}{\textbf{Điểm tổng quy đổi về thang 10}}&\multicolumn{5}{|p{0.205\linewidth}|}{} \\
    \hline
    \end{tabular}
    \label{mul_table}
\end{table}
\newpage
\raggedright\textbf{\itshape\fontsize{13pt}{20pt}\selectfont Nhận xét khác của cán bộ phản biện}
\newline


\vspace{5cm}
\hspace{9cm}Hà Nội, ngày\hspace{0.5cm}tháng\hspace{0.5cm}năm

\hspace{10cm}\textbf{Người nhận xét}
\vspace{2cm}
\hspace{9.5cm} (Ký và ghi rõ họ tên)
