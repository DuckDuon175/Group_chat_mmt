
\section*{PHẦN MỞ ĐẦU}
\phantomsection\addcontentsline{toc}{section}{\numberline{} PHẦN MỞ ĐẦU}
\subsection*{Đặt vấn đề}

% \addcontentsline{toc}{section}{\numberline{} Đặt vấn đề}
% Cuộc cách mạng công nghiệp lần thứ tư (hay còn được gọi là cuộc cách mạng công nghiệp 4.0) đã và đang phát triển với tốc
% độ rất nhanh, ảnh hưởng đến mọi mặt đời sống xã hội. Nội dung cốt lõi của cuộc cách mạng chính là sự kết hợp giữa 
% khoa học công nghệ, trí tuệ nhân tạo và sự sáng tạo của con người. Đối với Việt Nam đang trong quá trình công nghiệp hoá, 
% hiện đại hoá, việc áp dụng được những công nghệ mới trong một số lĩnh vực thiết yếu của xã hội, đặc biệt trong ngành y tế, 
% chính là nền tảng quan trọng để chăm sóc sức khoẻ con người, từ đó tạo nên những con người với sức khoẻ tốt nhất, sẵn sàng 
% đóng góp cho sự phát triển của đất nước. 
Lĩnh vực y tế đang có những bước chuyển mình lớn trong cuộc cách mạng công nghiệp lần thứ tư 
(hay còn được gọi là cuộc cách mạng công nghiệp 4.0). Đại dịch COVID-19 đã chứng minh được tầm quan trọng của việc áp dụng
khoa học kỹ thuật vào những sản phẩm y tế giúp đẩy lùi dịch bệnh,
có thể kể đến như máy rửa tay tự động do TS.Hàn Huy Dũng (đang công tác tại Trường Điện - Điện tử, thuộc Đại học Bách khoa Hà Nội) 
\cite{ref_thay_dzung} cùng các cộng sự sáng chế, và một số ứng dụng di động
nổi bật được hầu hết người dân Việt Nam sử dụng trong đại dịch COVID-19 như Bluezone - ứng dụng cảnh báo tiếp xúc gần với
những người nhiễm COVID qua Bluetooth low energy, ứng dụng NCOVI, theo dõi các ca nhiễm và thực hiện khai báo y tế, ứng
dụng PC-Covid để cập nhật các thông tin tiêm vắc xin, thông tin xét nghiệm. Đây là một tín hiệu cho thấy nước ta đang áp
dụng công nghệ 4.0 vào trong ngành y tế một cách chủ động. Hiện nay, việc chăm sóc sức khoẻ đang được chú trọng, đối với 
những những người muốn tự theo dõi sức khoẻ bản thân
thì các thiết bị nhỏ gọn có chức năng hỗ trợ đo và một ứng dụng di động hỗ trợ việc theo dõi sức khoẻ là một trong những
điều quan trọng và cần thiết. Kể cả đối với những người gặp vấn đề về sức khoẻ, việc đến bệnh viện đông đúc để theo dõi
và khám là khá khó khăn. Câu hỏi đặt ra là có phương án khả thi nào có thể áp dụng cho việc chăm sóc sức khoẻ một cách cụ thể, 
người dùng vẫn có thể theo dõi sức khoẻ tại nhà, đồng thời có sự cố vấn và hỗ trợ của những người có chuyên môn không?

\subsection*{Đề xuất hệ thống}
% \phantomsection\addcontentsline{toc}{section}{\numberline{} Đề xuất hệ thống}

Trong mục Đặt vấn đề, chúng em có trình bày về vấn đề chăm sóc sức khoẻ tại nhà, ở mục này, chúng em xin phép đề xuất một hệ thống
dựa trên những tiến bộ của khoa học kĩ thuật, với nền tảng là sự kết hợp giữa thiết bị IOT cùng hệ thống ứng dụng theo dõi, lưu trữ,
giúp tiết kiệm thời gian cũng như tăng khả năng tiếp cận cho nhiều người dùng. 

Với nền tảng được tiếp cận thiết bị đo điện tim của mẹ và thai nhi bằng điện cực không tiếp xúc trong thời gian gần đây, 
cùng với đó là việc phát triển các thiết bị IOT đang rất được chú trọng hiện nay, chúng em mong muốn xây dựng được một hệ thống
có thể kết nối được các thiết bị đo điện tim, thu thập dữ liệu điện tim thời gian thực, đồng thời có thể lưu trữ và phục vụ cho mục đích
phân tích dữ liệu về sau này. Cụ thể, hệ thống chúng em đề xuất sẽ gồm:

\begin{adjustwidth}{1.5em}{}
  \begin{itemize}
      \item Một ứng dụng di động cho riêng những người dùng để có thể kết nối với thiết bị đo điện tim để theo dõi trực tiếp tình trạng sức khoẻ, đồng
      thời có thể trao đổi với bác sĩ về tình trạng sau các lần đo, xem tin tức về các thông tin liên quan tới sức khoẻ
  
      \item Một ứng dụng di động cho bác sĩ để có thể xem được kết quả đo của các bệnh nhân được quản lý, trao đổi được với bệnh nhân
  
      \item Một ứng dụng web cho admin để quản lý hệ thống, đặc biệt là có phần phân công bác sĩ quản lý cho bệnh nhân
  
      \item Một server để lưu cơ sở dữ liệu liên quan đến người dùng và dữ liệu đo của người dùng, có thể phục vụ cho công tác nghiên cứu và
      phân tich dữ liệu sau này.
  \end{itemize}
  \end{adjustwidth}


\subsection*{Mục tiêu của đề tài}
Sau khi đã trình bày đề xuất về một hệ thống theo dõi và quản lý dữ liệu điện tim, mục tiêu chúng em muốn đạt được khi
làm đề tài này đó là:

\begin{adjustwidth}{1.5em}{}
  \begin{itemize}
      \item Nắm được cơ sở lý thuyết và cách thiết kế, ứng dụng các thiết bị trong hệ thống IOT

  
      \item Thực hiện hoàn chỉnh các ứng dụng được đề ra trong mục Đề xuất hệ thống, các ứng dụng hoạt động ổn định

  
      \item Có thể kết hợp tốt với các thiết bị phần cứng đang được hợp tác nghiên cứu

  
      \item Cung cấp tài liệu tham khảo một cách đầy đủ, trung thực.

  \end{itemize}
  \end{adjustwidth}





\subsection*{Phương pháp nghiên cứu}
% //TODO: hoàn thành phần phương pháp nghiên cứu
Trong đồ án lần này, chúng em đã thực hiện kết hợp nhiều phương pháp nghiên cứu

\subsection*{Kết quả đạt được}

Trong suốt quá trình thực hiện đồ án, hai chúng em Đồng Minh Thái, Hoàng Anh Tuấn đã được tìm hiểu và nghiên cứu sâu hơn về cả phần cứng,
hệ thống IOT, cách kết nối các hệ thống với nhau. Các kết quả đạt được cho đến thời điểm hoàn thiện quyển đồ án bao gồm:

\begin{adjustwidth}{1.5em}{}
  \begin{itemize}
      \item Hoàn thành quyển đồ án với nội dung chi tiết về quá trình xây dựng và phát triển hệ thống

  
      \item Hoàn thành các sản phẩm ứng dụng đã đề ra trong mục Đề xuất hệ thống, các sản phẩm đã có sự kết nối, dữ liệu điện tim được theo dõi
      thời gian thực được lưu trên server, dữ liệu có thể được phân tích và nghiên cứu sau này
  
      \item Được phát triển các kỹ năng làm việc nhóm, viết đồ án, kết hợp với team phần cứng, team firmware để các sản phẩm được hoàn thiện hơn.

    \end{itemize}
  \end{adjustwidth}




\subsection*{Cấu trúc đồ án}
% \phantomsection\addcontentsline{toc}{section}{\numberline{} Cấu trúc đồ án}

\begin{itemize}
  \item Phần mở đầu: Trình bày về mục đích của đồ án, thu thập yêu cầu, đề xuất hệ thống, phần tích tính khả thi và bố cục đồ án.
  \item Chương 1: Trình bày chi tiết các khâu trong phân tích hệ thống. 
  Bao gồm xác định yêu cầu, thiết kế sơ đồ use case, biểu đồ hoạt động, biểu đồ tuần tự.
  \item Chương 2: Trình bày chi tiết khâu thiết kế cho hệ thống. Bao gồm thiết kế giao diện phần mềm, thiết kết API, thiết kế cơ sở dữ liệu
  và giải pháp tối ưu hiệu nặng.
  \item Chương 3: Trình bày khâu triền khai và kiểm thử.
  \item Chương 4: Kết luận và nêu ra hướng phát triển.
 
\end{itemize}


\cleardoublepage

% \pagenumbering{arabic}